% vi: spell spl=en
%\documentclass[11pt,a4paper,oldtoc,final]{article}
\documentclass[11pt,a4paper,oldtoc,final]{tufte-book}
%\usepackage{a4wide}

\usepackage{float}
\usepackage{subfig}
\usepackage{graphicx}
\usepackage{makeidx}
\makeindex

\usepackage{mathptmx}
\usepackage[T1]{fontenc}

%\usepackage{bookman}
%\usepackage[T1]{fontenc}

\title{ML Cookbook}
\author{Frans Slothouber}
\publisher{We Dig Data}
\date{\today}

\newcommand{\blankpage}{\newpage\hbox{}\thispagestyle{empty}\newpage}
% Prints the month name (e.g., January) and the year (e.g., 2008)
\newcommand{\monthyear}{%
  \ifcase\month\or January\or February\or March\or April\or May\or June\or
  July\or August\or September\or October\or November\or
  December\fi\space\number\year
}

%=============================================================================

\begin{document}

%=============================================================================
\frontmatter
\blankpage

\maketitle

% copyright page
\newpage
\begin{fullwidth}
~\vfill
\thispagestyle{empty}
\setlength{\parindent}{0pt}
\setlength{\parskip}{\baselineskip}
Copyright \copyright\ \the\year\ \thanklessauthor

\par\smallcaps{Published by \thanklesspublisher}

% \par\smallcaps{\url{http://wedigdata.com}}

\par This work is licensed under the Creative Commons Attribution-ShareAlike
4.0 International License. To view a copy of this license, visit
\url{http://creativecommons.org/licenses/by-sa/4.0/}. \index{license}

\par\textit{First printing, \monthyear}
\end{fullwidth}

% contents
\tableofcontents

\listoffigures

\listoftables




\chapter*{Introduction}

In the book we take a single dataset and use this to 
exhaustively try out a large number of machine learning techniques.


%=============================================================================
\mainmatter

\chapter{The Dataset}

The dataset used in this book is a synthetic one.
It has data on fictional owl species: forest owls and city owls.
\index{city owl} \index{forest owl}
The full dataset contains $100,000$ samples and is divided by uniform sampling
into an equally sized test and training set.

\begin{marginfigure}
    \includegraphics[width=\textwidth]{Figures/Correlations/cor}
    \caption{Relations between the features}
\end{marginfigure}

The set contains the following features:
\begin{itemize}
    \item kind     - a factor with two levels: forest, city.
    \item gender   - a factor with two levels: male, female.
    \item eyesize  - number
    \item headsize - number
    \item age      - number
    \item height   - number
    \item wingspan - number
    \item weight   - number
\end{itemize}


\begin{figure*}
    \subfloat{\includegraphics[width=0.45\textwidth]{Figures/DatasetStats/age_density}}
    \subfloat{\includegraphics[width=0.45\textwidth]{Figures/DatasetStats/weight_density}} \\
    \subfloat{\includegraphics[width=0.45\textwidth]{Figures/DatasetStats/height_density}}
     \caption{Density plots for the various factors in the dataset}
\end{figure*}


\chapter{Training}

\section{Random Forest}

\cite{Breiman2001}

\chapter{Validation}

\chapter{Sandbox}

Test \cite{Tufte2006}

% Random Forest
% SVM 

% Too Much Data
% Too Little Data
% Too Many Features
% Missing Data

% R function for given tasks
%
% Scaling / normalization  scale()

% Terms

% Factor
% Feature
% Validation set
% Test set
% Training set
% Model
% CV
%

\backmatter

\bibliography{book}
\bibliographystyle{plainnat}

\printindex

\end{document}



